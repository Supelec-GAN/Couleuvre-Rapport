\chapter{Réseaux de neurones}

\section{Problème et de classification}
Dans la théorie de l'apprentissage statistique, la classification a pour objectif de déduire d'un nombre fini d'observations indépendantes une partition de l'espace en un ensemble a priori inconnu de domaines de l'espace appelés classes. 

%%%%%%%%%%%%%%%%%%%%%%%%%%%%%%%%%%%%%%%%%%%%%%%%%%%%%%%%%%%%%%%%%%%%%%%%%%%%%%%%%%%
%                               A COMPLETER                                       %
%%%%%%%%%%%%%%%%%%%%%%%%%%%%%%%%%%%%%%%%%%%%%%%%%%%%%%%%%%%%%%%%%%%%%%%%%%%%%%%%%%%

\section{Apprentissage supervisé}

%%%%%%%%%%%%%%%%%%%%%%%%%%%%%%%%%%%%%%%%%%%%%%%%%%%%%%%%%%%%%%%%%%%%%%%%%%%%%%%%%%%
%                               A COMPLETER                                       %
%%%%%%%%%%%%%%%%%%%%%%%%%%%%%%%%%%%%%%%%%%%%%%%%%%%%%%%%%%%%%%%%%%%%%%%%%%%%%%%%%%%

\section{Structure d'un neurone}

L'extraordinaire capacité d'apprentissage et d'adaptation des réseaux de neurones biologiques a poussé les scientifiques à tenter de modéliser informatiquement leur fonctionnement afin d'exploiter ces capacités. 

Les observations biologiques ont mené au premier modèle du neurone, divisé en 3 parties. Le neurone reçoit des signaux chimio-électriques par ses dendrites; ces signaux sont traités dans le corps cellulaire, où un effet de seuil est appliqué; enfin l'axone permet la transmission du signal chimio-électrique de sortie. De nouvelles découvertes sur les neurones ont permis de complexifier grandement ce modèle, mais c'est celui qui sert de base aux neurones artificiels. \\

\begin{definition}[Neurone] 
Un neurone à $n$ entrées permet de modéliser une fonction de $\R^n$ dans $\R$. Un neurone est défini par la données de trois paramètres :
  \begin{itemize}
    \item un vecteur de poids $\omega \in \R^n$
    \item un biais $b \in \R$
    \item une fonction d'activation $g \in \R^\R$
  \end{itemize}

La fonction $f$ modélisée par le neurone s'écrit alors : 
  \begin{equation}
\forall x \in \R^n, f(x) = g(\omega^{T}x - b) = g(\sum_{i=0}^{n-1}\omega_{i}x_{i} - b)
  \label{neuron_function_equation}
  \end{equation}
\end{definition}    
\begin{remark}[Biais]
Il est possible de rajouter une composante $x_{n} = -1$ à tous les vecteurs d'entrées afin de pouvoir considérer le biais $b$ comme la composante $\omega_n$ du vecteur de poids. L'équation \eqref{neuron_function_equation} devient alors
  \begin{equation}
\forall x \in \R^n, f(x) = g(\omega^{T}x) = g(\sum_{i=0}^{n}\omega_{i}x_{i})  
  \label{corrected_neuron_function_equation}
  \end{equation}

Cependant, considérer le biais comme un poids pose des problèmes lorsqu'on utilise l'approche des graphes de calculs, qui sera explicités dans la suite de ce rapport. Nous n'avons donc pas appliqué cette simplification pour nos propres réseaux.

\end{remark}

%%%%%%%%%%%%%%%%%%%%%%%%%%%%%%%%%%%%%%%%%%%%%%%%%%%%%%%%%%%%%%%%%%%%%%%%%%%%%%%%%%%
%                               A COMPLETER                                       %
%%%%%%%%%%%%%%%%%%%%%%%%%%%%%%%%%%%%%%%%%%%%%%%%%%%%%%%%%%%%%%%%%%%%%%%%%%%%%%%%%%%


\section{Réseaux de neurones}

Les possibilités limitées d'un neurone seul ont donc poussé l'informatique à agencer les neurones en réseaux afin de créer des modèles plus complexes. \\

\begin{definition}[Réseau de neurone - Perceptrons] 
Un réseau de neurones est défini par un graphe orienté $\mathcal{G}(V, A)$ où les n\oe{}uds sont des neurones et les arêtes des liens entre les neurones et par un ensemble de neurones d'entrée $V_{in} \subset V$ et de neurones de sorties $V_{out} \subset V$. Une arête partant d'un neurone $i$ vers un neurone $j$ signifie que la sortie du neurone $i$ est une entrée pour le neurone $j$. Notons $f_{j}$ la fonction représentant le neurone $j$. Finalement, nous noterons un réseau $\mathcal{N}(V, A, V_{in}, V_{out}, f)$ où $f$ est l'ensemble des fonctions des neurones.\\
\end{definition}

\begin{definition}[Perceptron]
Un réseau de neurones est de type perceptron lorsque le graphe qui le représente est acyclique (réseau feedforward) et qu'il peut être représenté comme une suite finie de couches de neurones telle que les sorties d'une couche sont exactement les entrées de la couche suivante.
\end{definition} 

\section{Propagation dans un perceptron}

On cherche à obtenir les équations qui régissent la propagation des entrées dans un réseau perceptron. Comme les neurones sont agencées par couches, on peut indicer chaque neurone par son numéro de couche et sa position dans la couche. De plus, on parlera dans la suite indistinctement du neurone ou de la fonction qu'il modélise. 

Alors, soit un réseau perceptron à $N$ couches. Pour $i \in \brackets{1,\ N}$ soit $n_i$ le nombre de neurones de la couche $i$. Pour tout $i,j $ tel que $i \in \brackets{1,\ N}, j \in \brackets{1,\ n_i}$, le neurone $(i,j)$ est défini par 
\begin{itemize}
  \item sa fonction d'activation $f_{i,j}$ 
  \item son vecteur de poids $\omega_{i,j}$
  \item son biais $b_{i,j}$
\end{itemize}

Comme on est dans un réseau perceptron, tous les neurones de la couche $i-1$ sont connectés au neurone $(i,j)$. Par conséquent le vecteur d'entrée de ce neurone est un vecteur $X_n$ de dimension $n_{i-1} \times 1$. Alors la sortie du neurone $(i,j)$ est égale à $f_{i,j}(\omega_{i,j}X_{i-1}^T - b_{i,j})$. En écrivant $X_n = \begin{pmatrix}X_0^n & ... & X_{n_i-1}^n\end{pmatrix}^T$, on a alors la relation de récurrence 
\begin{equation}
  X_i^{j+1} = f_{j,i}(\omega_{j,i}X_j^T - b_{j,i})
  \label{equation_propagation_coefficients}
\end{equation}

Dans un réseau perceptron, on suppose de plus que toutes les fonctions d'activations des neurones d'une même couche sont identiques. On a alors $\forall i \in \brackets{1,\ N},\ \forall j \in \brackets{1,\ n_i},\ f_{i,j} = f_i$
On peut alors réécrire l'équation \eqref{equation_propagation_coefficients} sous forme matricielle. On pose

\begin{align*}
\forall i \in \brackets{1,\ N},\ F_i &= \begin{pmatrix} f_i & ... & f_i \end{pmatrix}\quad \text{(Fonction d'activation de la couche)}\\
                            W_i &= \begin{pmatrix} \omega_{i,0} \mid & ... & \mid \omega_{i, n_i-1}\end{pmatrix}^T \quad \text{(Matrice des poids de la couche)} \\
                            B_i &= \begin{pmatrix} b_{i, n_i-1} & ... & b_{i, n_i-1}\end{pmatrix}^T 
\end{align*}

En nommant $X_0$ le vecteur en entrée du perceptron, on a la nouvelle équation matricielle : 
\begin{equation}
  \forall i \in \brackets{1,\ N},\ X_i = F_i(W_i X_{i-1} - B_i) 
  \label{equation_propagation_matricielle}
\end{equation}

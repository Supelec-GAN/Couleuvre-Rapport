\usepackage[T1]{fontenc}
\usepackage[utf8]{inputenc}
\usepackage{lmodern}
\usepackage{graphicx}
\usepackage{amsmath}
\usepackage{amssymb}
\usepackage{amsthm}
\usepackage{french}

\usepackage[left=3cm, right = 3cm, top = 3cm]{geometry}


% Facilite l'écriture de paragraphes de définitions, grâce à la commande \begin{definition}
\theoremstyle{definition}
\newtheorem{definition}{Définition}
\newtheorem{conjecture}{Conjecture}

% Facilite l'écriture de paragraphes de remarques et d'exemples, grâce à la commande \begin{example}
\theoremstyle{remark}
\newtheorem{remark}{Remarque}
\newtheorem{example}{Exemple}

% Symbole R des réels 
\def\R{\textrm{I\kern-0.21emR}}

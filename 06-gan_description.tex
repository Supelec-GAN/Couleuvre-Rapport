\chapter{Generative Adversarial Networks}

Après avoir appris à manipuler correctement les réseaux perceptrons simples, nous nous sommes intéressés à la structure de \textbf{Generative Adversarial Networks}, qui a constitué le coeur du projet, et son enjeu majeur. 

La naissance du GAN se place dans un contexte de recherche de moyens innovants et efficaces de génération de données arbitraires. Dans l'ère de l'information, les données constituent une ressource fondamentale, et la capacité d'en générer facilement de nouvelles représente un pan entier de la recherche, si ce n'est plusieurs. Le réseaux GAN sont une solution à base de réseaux à apprentissage semi-supervisés, dont l'objectif est de générer des données arbitraires en extrapolant à partir de données initialement fournies.

Dans ce chapitre, nous décrirons qualitativement le fonctionnement attendu des GANs et leur théorie, puis nous détaillerons les aspects plus techniques et mathématiques du problème. 

\section{Description du fonctionnement des réseaux GAN}
\subsection {L'article de Ian Goodfellow, la naissance du GAN}
\subsection{Les objectifs du GAN}
\subsection{Description qualitative du fonctionnement du GAN}

\section{Discriminateur et générateur}
\subsection{Théorie des jeux et équilibre de Nash}
\subsubsection{$D(G(z)) = D(x) = \frac{1}{2}$ : un premier critère d'optimalité}
\subsection{Les différentes distributions statistiques impliquées}
\section{es fonctions de coût, l'apprentissage}
\subsection{Les différentes fonctions de coût}
\subsection{Apprentissage par récompense ou par punition ?}
\subsection{Quand un réseau prend le pas sur l'autre}

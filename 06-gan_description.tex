\chapter{Generative Adversarial Networks}

Qu'est ce que le GAN ?
\section{Description du fonctionnement des réseaux GAN}«»
\subsection {L'article de Ian Goodfellow, la naissance du GAN}
\subsection{Les objectifs du GAN}
\subsection{Description qualitative du fonctionnement du GAN}«»

\section{Discriminateur et générateur}
\subsection{Théorie des jeux et équilibre de Nash}
\subsubsection{$D(G(z)) = D(x) = \frac{1}{2}$ : un premier critère d'optimalité}
\subsection{Les différentes distributions statistiques impliquées}
\section{es fonctions de coût, l'apprentissage}
\subsection{Les différentes fonctions de coût}
\subsection{Apprentissage par récompense ou par punition ?}
\subsection{Quand un réseau prend le pas sur l'autre}
